\subsection{Beneficio económico}

Solar Link propone el uso de un sistema solar off-grid de baja potencia, el cual es la alternativa mas económica que se consigue de energía solar. Este sistema consta de panel/es solar/es, batería/s, nuestro cargador MPPT de 3 etapas y un inverter. \\

El sistema que recomendamos instalar en un hogar promedio en Argentina, se compone de dos paneles solares de 380W, un inverter de 2000W, dos baterías de ciclo profundo de 110Ah, el cargador MPPT de 3 etapas de Solar Link, y el módulo Solar Link, con un costo total aproximado de esto ARS \$1.600.000 (USD \$1.600).\\

Mientras tanto, un sistema comparable on-grid, puede salir ARS \$5.000.000 (USD \$5.000). Esto depende del servicio que se contrate y de la empresa que lo realice.\\

No está de más mencionar que al uno producir la energía  que consume, se disminuye el consumo de energía de terceros, el del proveedor eléctrico. Esto quiere decir, que todos los meses se verá reducido el precio que hay que pagar por el suministro de energía eléctrico.\\

El ahorro en este caso no es lineal. Si uno termina consumiendo a fin de mes la mitad de energía que consumiría normalmente, por como funciona el sistema de cobros de las empresas distribuidoras en Argentina, no pagaría la mitad, sino que bastante menos.\\

Esto ocurre porque las empresas establecen un precio para cada KW/h dependiendo de la escala en donde se consuma. Por ejemplo, entre 0KW/h y 200KW/h tiene un precio asignado, entre 200KW/h y 400KW/h otro mayor, y si se exceden esos 400KW/h otro inclusive mayor (estas escalas dependen de la empresa a cargo del suministro eléctrico).\\

Osea, digamos, que cada KW/h entre 0KW/h y 200KW/h valga 1, entre 200KW/h y 400KW/h, 1,5, y si es mayor a 400KW/h, 2. si uno consume 600KW/h, estaría pagando 900, mientras que si uno consume 300KW/h, estaría pagando 350, bastante menos de la mitad. Este ejemplo ilustra como funciona el sistema de cobros de las compañías distribuidoras en Argentina. Las escalas y los valores son a modo de ejemplo. Pueden ser diferentes segun la empresa, la región, y el momento en donde se analice.\\

\subsection{Incentivo y concientización}

A través de nuestra aplicación, el usuario tendrá acceso a toda la información que respecta al consumo eléctrico de su casa. Esto le permite conocer al detalle no solo el consumo total, sino tambien el consumo que generó el sistema solar propio.\\

Esto genera en el usuario un sentido de ganancia y de gasto propio. Saber bien cuanto se esta gastando, porque y cuanto de este gasto se amortiguo ayuda a concientizar sobre el consumo de energia en general. \\

Mas allá del gasto de energía, tener un sistema que, ya sea de manera directa o indirecta, ayuda al medio ambiente, tiene un efecto positivo para la psiquis y para el entorno donde uno vive. Saber que uno esta aportando su grano de arena (siendo el mismo economicamente viable y rentable) genera tranquilidad y un sentido de responsabilidad ciudadana. \\

\subsection{Estado del arte}

Actualmente, en el mercado argentino, existen principalmente tres tipos de instalaciones de energía solar para un hogar. Cada tipo cuenta con sus propias ventajas y desventajas:\\

\begin{table}[H]
\begin{tabular}{|l|l|l|}
\hline
Tipo     & Ventajas                                                                                                                 & Desventajas                                                                                            \\ \hline
On-Grid  & \begin{tabular}[c]{@{}l@{}}Sistema acoplado a la línea doméstica\\ Alimenta casi el 100\% con energía solar\end{tabular} & \begin{tabular}[c]{@{}l@{}}Alto costo\\ Desperdicio de energía\end{tabular}                            \\ \hline
Off-Grid & \begin{tabular}[c]{@{}l@{}}Económico\\ Acumula energía\end{tabular}                                                      & \begin{tabular}[c]{@{}l@{}}Requiere una línea exclusiva \\ para este sistema\end{tabular}              \\ \hline
Híbrido  & \begin{tabular}[c]{@{}l@{}}Sistema acoplado a la línea doméstica\\ Acumula energía\end{tabular}                          & \begin{tabular}[c]{@{}l@{}}Potencia limitada\\ Requiere instalación especial\end{tabular} \\ \hline
\end{tabular}
\end{table}

Los sistemas \textbf{on-grid}, como su nombre lo indica, se instalan directamente en la línea de la casa. Toda la energía que producen los paneles se entregan a la línea general de la casa, sin almacenarse en ningún lado. Si la casa consume mas que lo que generan los paneles, se consume la energía correspondiente a este exceso del proveedor eléctrico, y si consume menos, la energía generada por demás se entrega a la red eléctrica del proveedor. \\

Para este tipo de sistemas es necesaria la instalación de un medidor bidireccional, que mide tanto el consumo de la casa como lo que devuelve por el exceso en la producción. Si bien este medidor descuenta lo que uno devuelve a la red eléctrica, lo que uno consume es mas caro, no es muy redituable y no su instalación depende del proveedor de energía eléctrica.\\

Los sistemas \textbf{off-grid}, por otra parte, acumulan la energía generada por los paneles solares en baterías, para luego convertirla en los 220Vac que necesita la casa. El problema de esto, es que la energía extraída de las baterías no se puede incorporar directamente a la línea de la casa, por lo que es necesario la utilización de una línea exclusiva para lo que este sistema genere. Esto implica tener que hacer una instalación eléctrica aislada e independiente del resto de la casa, algo impráctico si además se tiene en cuenta que la utilización de esta línea debe ser controlada y depende directamente de la carga de los paneles.\\

Los sistemas \textbf{híbridos} son sistemas solares off-grid que se incorporan a la línea de la casa. Pueden utilizar tanto la energía solar acumulada como la energía del proveedor eléctrico. Para utilizar este tipo de sistema es necesaria una instalación especial, que se tiene que realizar en líneas dedicadas. Además, por como es su principio de funcionamiento, el sistema solar no es expandible, por lo que la potencia instalada en un principio no podrá modificarse.\\

Mientras tanto, \textbf{Solar Link} busca recopilar todas las ventajas de estos 3 tipos de sistemas solares, y dejar en el camino todas las desventajas posibles. Se podría decir que Solar Link es un sistema que acumula y utiliza energía solar, acoplado a la línea de la casa, económico, modular expandible y amigable con el usuario, compartiendo toda la información necesaria sobre el sistema solar instalado y el consumo de la casa.
